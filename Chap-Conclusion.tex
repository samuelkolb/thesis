%!TEX root=Thesis.tex
\chapter{Conclusion}
\label{cha:conclusion}

The research in this thesis has focused on solving two problems: learning constraints and learning optimization criteria.
Implementation for both clause learning and clausal optimization have been provided that accomplish these tasks using first order logic clauses.

\paragraph{Learning constraints}
The first goal of this thesis was to learn constraints that describe a given set of examples.
The clause learning system has been shown to be able to fulfill this task accurately and efficiently.
It has been able to identify hard constraints as well as soft constraints which describe the underlying problem.
Input to the system has been restricted to information that is easily provided by a user, such as typing information and examples.
However, the use of logical solver also enables users to provide rich background knowledge.

The influence of external factors, such as the search parameters and the number of examples have been measured to provide insights on their effects.
Various design decisions have been experimentally validated.

This research demonstrates that logical clauses and ILP techniques can be used successfully to learn constraints.
Contrary to many other constraint learning systems, the learned constraints are independent of the specific examples and domains.
It has previously been stated that in constraint learning typically few examples are available
Often solving these problems is non trivial which limits the ability of the user to produce positive examples.
Domain independent constraints, however allow the positive examples to be of a different size and complexity.
For example, providing a solved $4 \times 4$ sudoku is much a easier than a $9 \times 9$ sudoku.

\paragraph{Learning optimization criteria}
The second goal of this thesis was to learn optimization criteria based on rankings over examples.
The clausal optimization system was able to successfully learn weighted clauses that are able to model user preferences.
Compared to absolute scores, rankings are typically easier to provide for a user.
Experiments have shown that the system is able to deal with conflicting rankings.
Even for few examples, optimization criteria are found that can identify the optimal solution.

The success of the clausal optimization system shows that the suggested representation of weighted clauses can be used to model optimization criteria.

\paragraph{Solving}
\begin{figure}

	\caption{From problem to solution}
	\centering
		\includegraphics[width=0.9\textwidth]{ProblemToSolution.pdf}
	\label{fig:problem_to_solution2}

\end{figure}

As originally stated in the introduction these learning systems aim to facilitate and automate the task of generating formal representations of problems.
This forms the first step of the process illustrated in figure~\ref{fig:problem_to_solution2}.
Given such a representation, a constraint solver is used to generate solutions.
Solvers such as IDP can use hard constraints in the form of clauses to generate solutions.
However, the weighted first order logic clauses that are used as optimization criteria in this thesis, cannot be used directly.
The problem of identifying optimal solutions for such optimization criteria can be seen as a first order extension of the MAX-SAT problem.
In this thesis it has been outlined how the knowledge base system IDP can be used to generate optimal solutions for weighted clauses in first order logic.
The approach also allows more expressive constraints to be used.
This should encourage the exploration of the automatic acquisition of more expressive relational constraints.
Contrary to typical settings, the weights of the soft constraints do not have to be fixed, they can be dependent on the values of the variables in a solution.

\paragraph{Future work}
This thesis spawns several interesting opportunities for future work.
Several of these will be briefly described.
\\\\
The number of variables and literals form transparent but unintuitive parameters.
Typically, a user will not possess any knowledge about these parameters.
It would be interesting to use negative examples and heuristics to automatically adapt these parameters and realize a so called shift of bias during the search.
Generally, negative examples can be incorporated to ensure that theories are specific enough to discriminate between positive and negative examples.
This offers the possibility to define a stopping criterion.
It should not be hard to add support for negative examples, the Claudien system, for example, already had the capability to use negative examples.
\\\\
Clauses, in the current implementation, either cover an example or not.
Soft constraints must cover some of the examples.
Therefore, if an example consists of a large graph, then the constraint must hold everywhere in that graph to cover that example.
It would be interesting to add the ability to reason about \emph{substitutions}.
A substitution is an assignment of values to all variables in a clause.
Many such substitutions can occur in a large graph.
Thresholds for soft constraint within a simple example could then be defined in terms of substitutions.
This would also benefit the clausal optimization algorithm, which could use these soft constraints.
\\\\
Adding interactivity to the clause learning or clausal optimization process can also provide numerous benefits.
By using the output of the learning systems to generate new examples, the problem of having few examples can be alleviated.
The combination with (partial) queries as they are used in \cite{bessiere2013constraint,bessiere2007query} could be explored.
Active learning has been proposed in a propositional setting \cite{campigotto2011active} for learning optimization criteria.
For clausal optimization the rankings could be also generated interactively.
The system could automatically determine what rankings would provide the most information and query the user.
\\\\
Several additions to the current input format could be beneficial.
The symmetric predicate has been illustrated as an intuitive way to specify background knowledge that can help speed up clause learning process.
It would be interesting to allow other types of background to be expressed in a similar way.
Additionally, the learned clauses are currently independent of specific objects.
This is useful as this allows more flexibility.
However, the refinement operator could be extended to include the use of global constants, which themselves are independent of specific examples.
\\\\
The learn to rank system that is used by the clausal optimization process ignores equalities.
However, equalities can be very expressive.
More research should be performed to determine if these would be a useful addition and how they could be incorporated.