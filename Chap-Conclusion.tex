%!TEX root=Thesis.tex
\chapter{Conclusion}
\label{cha:conclusion}

The research in this thesis has focussed on solving two problems: learning constraints and learning optimization criteria.
Implementation for both clause learning and clausal optimization have been provided that accomplish these tasks using first order logic clauses.

\paragraph{Learning constraints}
The first goal of this thesis was to learn constraints that describe a given set of examples.
The clause learning system has been shown to be able to fullfill this task accurately and efficiently.
It has been able to identify hard constraints as well as soft constraints which describe the underlying problem.
Input to the system has been restricted to information that is easily provided by a user, such as typing information and examples.
However, the use of logical solver also enables users to provide rich background knowledge.

The influence of external factors, such as the search parameters and the number of examples have been measured to provide insights on their effects.
Various design decisions have been experimentally validated.

This research demonstrates that logical clauses and ILP techniques can be used succesfully to learn constraints.

\paragraph{Learning optimization criteria}
The second goal of this thesis was to learn optimization criteria based on rankings over examples.
The clausal optimization system was able to succesfully learn weighted clauses that are able to model user preferences.
Compared to absolute scores, rankings are typically easier to provide for a user.
Experiments have shown that the system is able to deal with conflicting rankings.
Even for few examples, optimization criteria are found that can identify the optimal solution.

The success of the clausal optimization system shows that the suggested representation of weighted clauses can be used to model optimization criteria.

\paragraph{Solving}
As originally stated in the introduction these learning systems aim to facilitate and automate the task of generating formal representations of problems.
This forms the first step of the process illustrated in figure~\ref{fig:problem_to_solution2}.
Given such a representation, a constraint solver is used to generate solutions.
Solvers such as IDP can use hard constraints in the form of clauses to generate solutions.
However, the weighted first order logic clauses that are used as optimization criteria in this thesis, cannot be used directly.
The problem of identifying optimal solutions for such optimization criteria can be seen as a first order extension of the MAX-SAT problem.
In this thesis it has been outlined how the knowledge base system IDP can be used to generate optimal solutions for weighted clauses in first order logic. 

\begin{figure}

	\caption{From problem to solution}
	\centering
		\includegraphics[width=0.9\textwidth]{ProblemToSolution.pdf}
	\label{fig:problem_to_solution2}

\end{figure}

\paragraph{Future work}
This thesis spawns several interesting opportunities for future work.
First of all, the number of variables and literals form transparant but unintuitive parameters.
Typically, a user will not possess any knowledge about these parameters.
It would be interesting to use negative examples and heuristics to automatically adapt these parameters and realize a so called shift of bias during the search.

Generally, negative examples can be incorporated to ensure that theories are specific enough to discriminate between positive and negative examples.
This offers the possibility to define a stopping criterion.
The Claudien system already had the capability to use negative examples.

Adding interactivity to the clause learning or clausal optimization process can also provide numerous benefits.
By using the output of the learning systems to generate new examples, the problem of having few examples can be alleviated.
Active learning has been proposed in a propositional setting \cite{campigotto2011active} and the combination with (partial) queries \cite{bessiere2013constraint,bessiere2007query} could also offer additional capabilities.

The symmetric predicate has been illustrated as an intuitive way to specify background knowledge that can help speed up clause learning process.
It would be interesting to allow other types of background to be expressed in a similar way.

Currently the learned clauses are independent of specific objects, which allows more flexibility.
However, the refinement operator could be extended to allow the use of global constants which are also independent of specific examples.

The learn to rank system that is used by the clausal optimization process ignores equalities.
However, equalities can be very expressive.
More research should be performed to determine if these would be a useful addition and how they could be incorporated.