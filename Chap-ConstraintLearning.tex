%!TEX root=Thesis.tex
\chapter{Constraint Learning}
\label{cha:cl}

% Concepts
% (Herbrand) Interpretation
% Clause
% Range restricted

%[TODO Write] Given examples we want to learn logical %constraints that describe the data and can be used by a %logical solver to generate models.

%+ We want to model user preferences to be able to find an optimal solution

\section{Problem Statement}
This thesis attempts to solve two problems: learning constraints from examples and learning user preferences to compare different solutions. The first problem is independent of the second.

\subsection{Learning constraints}
Consider a set \sym{D} of examples. Examples in \sym{D} are Herbrand interpretations, every example consists of a set of ground atoms. Given a language of clauses \sym{L}, the goal of the constraint learner is to find a complete set \sym{T} of maximally specific clauses $\in \sym{L}$, that cover the examples in \sym{D}.
\\\\
The language \sym{L} determines what clauses are considered to be included in \sym{T}. It captures the syntactic restrictions and the bias. In this case elements of \sym{L} are restricted to functor-free clauses. A bias of a system captures implicit or explicit constraints of the system. In this case only range-restricted clauses are considered.
\\\\
It can often be useful to include domain\,/\,background knowledge. This knowledge is an input to the learning system and has the form of a logical theory \sym{K}. Without background knowledge a clause \obj{c} covers an interpretation \sym{I} if it is a model of the clause: $\sym{I} \models \obj{c}$. In the case that background knowledge is used this becomes: $\sym{I} \models \obj{c} \land \sym{K}$
\\\\
A theory \sym{T} allows to discriminate interpretations that are models of the theory from the ones that are not. A logical constraint solver can use \sym{T} to generate interpretations that adhere to the theory.

\subsection{Evaluating constraints}
Given the output of the learning phase, this thesis will examine the question: can a theory \sym{T} of learned clauses be used successfully to model constraint problems? 
\\\\
Systems like IDP have been shown [Ref IDP-solving?] to be able to generate models and extend partial models given a logical theory. The success of learning constraints from examples will depend on the quality of the theory generated by the learning system.

\subsection{Learning user preferences}
[Temp Broadly - to be expanded later] A theory \sym{T} learned from examples can discriminate between valid and invalid interpretations. The goal of learning user preferences is to define optimization criteria that impose an ordering on valid interpretations.
\\\\
In the context of constraint solving, \sym{T} is seen as a specification of a constraint problem and valid interpretations form the solutions of that problem. An order on interpretations can be used to identify more desirable solutions.