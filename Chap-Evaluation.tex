%!TEX root=Thesis.tex
\chapter{Evaluation}
\label{cha:evaluation}

% \begin{table}[!htp]
% 	\begin{tabularx}{\textwidth}{r|llllllX}
% 	    \toprule
% 	    Circles & 7 & 8 & 9 & 10 & 11 & 12 & 13  \\
% 	    \midrule
% 	     30 seconds & 0.976 & 0.962 & 0.948 & 0.905 & 0.891 & 0.859 & 0.852\\
% 		 60 seconds & 0.976 & 0.977 & 0.944 & 0.957 & 0.908 & 0.902 & 0.915\\
% 		 90 seconds & 0.978 & 0.951 & 0.971 & 0.952 & 0.886 & 0.937 & 0.901\\
% 		120 seconds & 0.978 & 0.953 & 0.968 & 0.928 & 0.916 & 0.913 & 0.898\\
% 	    \bottomrule
% 	\end{tabularx}
% 	\label{tbl:lin-best}
% 	\caption{Best relative scores of linear method}
% \end{table}

Several experiments have been run in order to assess how well the implementation accomplishes the clause learning and clausal optimization tasks.
Moreover, the influence of different factors internal and external to the algorithm is evaluated.
The clause learning and clausal optimization workflows will be examined separately, in that order.

\section{Learning}

[!! Question 1, problem statement]

\subsection{Accuracy}

\subsection{Speed}

\subsection{Clause length / number variables / arity}

\subsection{Symmetric predicates}

\subsection{Subset test}

\subsection{Compare to human}

\section{Optimization}

\subsection{Accuracy}

\subsection{Size examples / preferences}

\subsection{Noise}

\subsection{Tradeoff}

\subsection{W/wo hard constraints}

\subsection{Expressiveness}

\section{Open}
?? Compare to others