%!TEX root=Thesis.tex
\chapter{Related Literature}
\label{cha:rellit}

In this chapter, different previous research will be explored.
On the one hand, the state of the art will be described.
On the other hand, research whose contributions this thesis builds on will be explained.

\section{Constraint Learning}
Learning constraints is a difficult problem.
Unlike typical machine learning problems, there is usually little data available from which to learn the constraints.
%While constraint learning is interesting in an automated setting, the goal of some approaches, including this thesis, is also to interact with a human user. This introduces additional difficulties. Human users often lack the skill to 
Different approaches have been taken to tackle this problem.

\subsection{Exisiting approaches}
\paragraph{Learning from queries}
[Info more information needed !] [Ref Quote Note ?]
In order to alleviate the problem of having little data, systems can choose to use queries and user input to guide the search.
[Rewrite "Note mentions" !] Conacq and Quacq [Ref Add refs !] follow this approach. 
Info Does conacq use queries? It uses version space.. ?]

\paragraph{Global constraints and structure}
In constraint programming, one often distinguishes between \textit{local} and \textit{global} constraints.
Local constraints are describe ``simple'' relationships between variables such as $x_1 \neq x_2$.
These constraints can usually not be decomposed.
On the other hand, global constraints describe sets of constraints such as $\mathit{alldifferent}(x_1, x_2, x_3)$.
This global constraint corresponds with a set of local constraints: $x_1 \neq x_2, x_1 \neq x_3, x_2 \neq x_3$.
Besides reducing the number of constraints to model a problem, these global constraints can often be used by a solver to find solutions more efficiently.

Model Seeker \cite{Beldiceanu:ModelSeeker} is an effective approach to constraint learning, which uses global constraints.
It uses generators that structure variables in different ways by arranging them in tensors.
Using a catalog of global constraints, it then tests what global constraints hold in these structures.
This generate-and-test approach has produced good results and works fast.

\paragraph{ILP for constraint Learning}
[Info more text needed !] [Info Introduce CPS etc ?]
One approach that uses parallels between ILP and constraint learning is \cite{Lallouet:LearningCP}.

The approach in this thesis is based on techniques from Inductive Logic Programming (ILP).
Learning systems in this domain often face similar challenges.
ILP attempts to find structure in examples and describe this structure using logical formulas.

\subsection{Clausal Discovery \& Claudien}
Claudien, a clausal discovery engine introduced in \cite{DeRaedt:ClausalDiscovery}, attempts to learn a definite clause theory from examples.
Alg. \ref{alg:cd} illustrates how the algorithm works on a high level. 

\begin{algorithm}
	\caption{The clausal discovery algorithm}
	\label{alg:cd}

	\begin{algorithmic}
	\State $\sym{D} \gets Examples$
	\State $Q \gets \{\square\}$
	\State $\sym{T} \gets \{\}$
	\While{$\lnot isempty(Q)$}
		\State $c \gets next(Q)$
		\If{$covers(c, \sym{D})$}
			\If{$\lnot entails(\sym{T}, c)$}
				\State $\sym{T} = \sym{T} \cup c$
			\EndIf
		\Else
			\State $Q \gets Q \cup \rho(c)$
		\EndIf
	\EndWhile
	\State $\sym{T} \gets prune(\sym{T})$
	\State \Return \sym{T}
	\end{algorithmic}
\end{algorithm}

The idea is to start from an empty clause ($\square$).
Clauses are generalized by adding an atom to either the head or the body of the clause in every step, this is the responsibility of the ideal and complete refinement operator $\rho$.
If a clause \obj{c} covers all the examples in \sym{D} and it is not logically entailed by the theory found so far then \obj{c} is added to the result set \sym{T}.
A clause \obj{c} is considered to cover an example \sym{I} iff \sym{I} is a model of \obj{c}: $\sym{I} \models \obj{c}$.
The clause \obj{c} is not refined because any clause $\obj{c_i} \in \rho^*(c)$ is a generalization of \obj{c} and thus logically entailed by \obj{c}.

[Add Bias, (explain why not dlab?) ?]

\subsection{Refinement operator}
The refinement operator plays a central role in clausal discovery, but also in different ILP problems.
General insights about refinement operators have been obtained from \cite{DeRaedt:LRLearning} and are described in chapter \ref{cha:bg}.
In \cite{DeRaedt:CondensedRepresentations} different strategies are discussed that can be used to improve refinement operators, amongst others the canonical form for clauses.

\section{Logical Constraint Solving}
\label{sec:logical_constraint_solving}
There are multiple constraint solvers such as Eclipse and Minizinc [!! Ref Eclipse, Minizinc ?], which use different constraint languages.
Since in this thesis constraints are expressed as logical clauses, it is useful to use a solver which can work directly with first order logic.
This will allow logical background knowledge to be used.

\subsection{IDP}
IDP [!! Ref IDP !] is a knowledge base system which supports an extension of first order logic called FO(.).
This extension includes aggregates, inductive definitions, and other powerful tools for specifying rich background knowledge.
Given a logical theory, IDP is able to perform multiple tasks which are important in this context.
It can check if a model is consistent with the theory, generate a model for a theory as well as expand a partial model.
Because IDP uses SAT solving technology, it is also able to offer the ability to calculate entailment between logical theories.
It is worth noting thath, currently, not all features of FO(.) can be used in entailment.
Performance wise IDP has proven to be amongst the state of the art in multiple answer set competitions [Ref competition !].
Since the system is being actively developed and supported, it is a good choice for a logical solver to be used in this research.

The main difference between a logical solver, like IDP, and traditional constraint solvers, is that the latter class of solvers supports global constraints, programming constructs such as loops and support for (real) numbers.
Global constraints allow constraint solvers to speed up their search and are very compact.
Since the logical constraint generation in this thesis does not focus on global constraints, however, they are not crucial for its implementation.
Additionally, global constraints can usually be emulated using logical background knowledge passed to the solver, if necessary.
Compactness and loops are useful for programmers, but since in this case the constraints are generated automatically these features are less important.
The implicit universal quantification of variables in logical clauses already allows reasoning over domains of variables.
Currently IDP can only reason with natural numbers and contains limited optimization capabilities.
This is its biggest disadvantage.
There is however, a large class of problems that do not require rational numbers.