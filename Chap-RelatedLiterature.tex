%!TEX root=Thesis.tex
\chapter{Related Literature}
\label{cha:rellit}

In this chapter, previous research will be explored.
For constraint learning different alternative approaches are introduced that attempt to solve a related problem and form the state of the art.
Previous research on clause learning is presented, which this thesis builds on.
The later sections introduce research that has led to software that is used in this research.

\section{Constraint Learning}
Learning constraints is a difficult problem.
Unlike typical machine learning problems, there is usually little data available from which to learn the constraints.
%While constraint learning is interesting in an automated setting, the goal of some approaches, including this thesis, is also to interact with a human user. This introduces additional difficulties. Human users often lack the skill to 
Different approaches have been taken to tackle this problem.

\subsection{Existing approaches}
Constraint learning is a broadly defined task.
Three different approaches are introduced that, passively or actively, learn constraints in different forms.
These approaches show different, promising ways of approaching the constraint learning problem.

\paragraph{Learning from queries}
In order to alleviate the problem of having little data, systems can choose to interactively query the user for knowledge.
Conacq2 \cite{bessiere2007query} and QuAck \cite{bessiere2013constraint} use these queries to guide their search.
Queries consist of assignments of values to the given variables, which are then approved or disproved by the user.
QuAck is also able to learn from partial assignments, where values are assigned to only some of the variables. 

\paragraph{Global constraints and structure}
In constraint programming, one often distinguishes between \textit{local} and \textit{global} constraints.
Local constraints are describe ``simple'' relationships between variables, such as $x_1 \neq x_2$.
These constraints can usually not be decomposed.
On the other hand, global constraints describe sets of constraints such as $\mathit{alldifferent}(x_1, x_2, x_3)$.
This global constraint corresponds with a set of local constraints: $x_1 \neq x_2, x_1 \neq x_3, x_2 \neq x_3$.
Besides reducing the number of constraints to model a problem, these global constraints can often be used by a solver to find solutions more efficiently.

Model Seeker \cite{Beldiceanu:ModelSeeker} is an effective approach to constraint learning, which uses global constraints.
It uses generators that structure variables in different ways, which corresponds to arranging them in tensors.
For every such structure the variables are then partitioned into disjoint, equally structured subsets.
Using a catalog of global constraints, the system finds constraints that hold on these subsets.

\paragraph{ILP for constraint Learning}
Within machine learning, inductive logic programming (ILP) is a research area concerned with learning logical descriptions.
Since relational logic can be used to express constraints ILP techniques can be used to learn such constraints.
Most of the constraint learning and ILP research is done independently, but a recent approach has attempted to use ILP techniques for constraint learning.

In \cite{Lallouet:LearningCP} a system is proposed to learn constraint in the form of first order logic formulas.
Instead of a top-down or bottom-up approach, bidirectional search is used to explore the search space.
While both positive and negative examples can be used, the system primarily uses negative examples for learning.
\\\\
This thesis also pursues the use of ILP techniques for constraint learning.
Clausal discovery is used to learn a theory of relational clauses, mainly from positive examples.
The search space is explored in a bottom-up manner.

\subsection{Clausal Discovery \& Claudien}
\label{sec:claudien}
Claudien, a clausal discovery engine introduced in \cite{DeRaedt:ClausalDiscovery}, attempts to learn a definite clause theory from examples.
Algorithm \ref{alg:cd} illustrates how the algorithm works on a high level. 

\begin{algorithm}
	\caption{The clausal discovery algorithm}
	\label{alg:cd}

	\begin{algorithmic}
	\State $\sym{D} \gets Examples$
	\State $Q \gets \{\square\}$
	\State $\sym{T} \gets \{\}$
	\While{$\lnot isempty(Q)$}
		\State $c \gets next(Q)$
		\If{$covers(c, \sym{D})$}
			\If{$\lnot entails(\sym{T}, c)$}
				\State $\sym{T} = \sym{T} \cup c$
			\EndIf
		\Else
			\State $Q \gets Q \cup \rho(c)$
		\EndIf
	\EndWhile
	\State $\sym{T} \gets prune(\sym{T})$
	\State \Return \sym{T}
	\end{algorithmic}
\end{algorithm}

The idea behind the algorithm is to start from an empty clause ($\square$).
Clauses are generalized by adding an atom to either the head or the body of the clause in every step, this is the responsibility of the refinement operator $\rho$.
If a clause \obj{c} covers all the examples in \sym{D} and it is not logically entailed by the theory found so far then \obj{c} is added to the result set \sym{T}.
A clause \obj{c} is considered to cover an example \sym{I} if and only if \sym{I} is a model of \obj{c}: $\sym{I} \models \obj{c}$.
The clause \obj{c} is not refined because any clause $\obj{c_i} \in \rho^*(c)$ is a generalization of \obj{c} and thus logically entailed by \obj{c}.

\paragraph{Refinement operator}
The refinement operator plays a central role in clausal discovery, but also in different ILP problems.
General insights about refinement operators have been obtained from \cite{DeRaedt:LRLearning} and are described in chapter \ref{cha:bg}.

In \cite{DeRaedt:CondensedRepresentations} different strategies are discussed that can be used to improve refinement operators, amongst others the canonical form for clauses.
Clauses are often not unique and a constraint can usually be written in different ways.
In order to avoid generating such redundant clauses, a canonical form can be used.
This form dictates that clauses must be structured in a certain way and clauses that do not follow this structure can be discarded.

\begin{example}
A possible rule concerning the structure of clauses containing predicates $p_1/1$ and $p_2/1$ could be, that $p_1/1$ must always occur before $p_2/1$ in the head and body.
Consider the clauses:
\cen{
	$c_1 = \mathit{false} \leftarrow p_1(x) \land p_2(x)$ \\
	$c_2 = \mathit{false} \leftarrow p_2(x) \land p_1(x)$.
}
If the refinement operator is complete, it can decide autonomously that $c_2$ is redundant, knowing that $c_1$ will be generated at some point.
\end{example}

By building clauses with respect to an ordering of the atoms occurring in it, the generation of redundant clauses can often be prevented.

\paragraph{Bias}
Claudien uses the $\sym{D}LAB$ bias specification language.
Such a bias consists of a pattern that specifies how clauses are built and which atoms can be added to a clause by the refinement operator.
Defining this bias is an important task and can be non trivial to a user.
Therefore, $\sym{D}LAB$ is not used in this thesis.

\section{Logical Constraint Solving}
\label{sec:logical_constraint_solving}
Given a set of constraints, constraint solvers are used to generate solutions.
Multiple constraint solvers are available, such as Eclipse \cite{apt2006constraint} and MiniZinc \cite{nethercote2007minizinc}.
Different constraint solvers use different constraint languages, which define the format in which constraints must be supplied to the program.
In this thesis the constraints are expressed as logical clauses.
While these can be translated into other constraints, it is useful to choose a solver that can work directly with first order logic formulas.
This will also allow general logical background knowledge to be used.
Background knowledge is used to provide additional information about predicates.

\subsection{IDP}
IDP \cite{de2013prototype,wittocx2008idp} is a knowledge base system which supports an extension of first order logic called $FO(.)$.
This extension includes aggregates, inductive definitions, and other powerful tools for specifying rich background knowledge.
Given a logical theory, IDP is able to perform multiple tasks (types of inference) which are important in this context.
It can check if a model is consistent with a theory (model checking), generate a model for a theory and expand a partial model (model expansion).
IDP also offers access to a theorem prover, thereby offering the ability to calculate entailment between logical theories.
Currently, however, not all features of $FO(.)$ can be used in entailment.

Performance wise IDP has proven to be amongst the state of the art in multiple answer set competitions (e.g. \cite{calimeri2014third}).
Moreover, the use of IDP as a front-end to SAT-solving has been examined in \cite{bruynooghe2014predicate}, where it was found to carry only a minor performance overhead compared to a standard SAT-solver.
The versatility of this knowledge base system makes it a good choice for a logical solver to be used in this research.
Additionally, the system is being actively developed and supported.

Traditional constraint solvers lack some of the capabilities of IDP, but offer support for global constraints, programming constructs - such as loops - and rational numbers.
Global constraints are compact, built-in constraints that allow constraint solvers to speed up their search.
Since the logical constraint generation in this thesis does not focus on global constraints, however, they are not crucial for its implementation.
Additionally, global constraints can usually be emulated using logical background knowledge passed to the solver, if necessary.
Compactness and loops are useful for programmers, but since, in this case, the constraints are generated automatically, these features are less important.
The implicit universal quantification of variables in logical clauses already allows reasoning over domains of variables.
Currently IDP can only reason with natural numbers and contains limited optimization capabilities.
This is its biggest disadvantage.
There is however, a large class of problems that do not require rational numbers.

\section{Optimization}
Optimization criteria are used to discriminate between better and worse solutions.
Soft constraints are constraints which do not have to be fulfilled, they are only true for some solutions.
These soft constraints can be assigned weights that express their desirability.
Such weighted soft constraints can be used as optimization criteria, by maximizing the sum of the weights of the satisfied constraints.

\paragraph{Learning weights}
The weighted MAX-SAT problem, described in section~\ref{sec:max_sat}, uses weighted propositional clauses as optimization criteria.
Research such as \cite{campigotto2011active}, have successfully automated the construction of utility functions and the learning of weights from user scores.
The optimization problems learned are, amongst others, MAX-SAT problems.
Inspired by this approach, this thesis attempts to learn weighted first-order logic clauses as optimization criteria.

\subsection{\svm}
\label{sec:svm}
Instead of using scores, the weights will be learned from rankings over example solutions.
In this process, \svm{} \cite{joachims2002optimizing} will be used to find optimal weights based on rankings.
Rankings are grouped together and can span an arbitrary amount of instances.
Equalities are ignored, as rankings are decomposed into sets of pairwise inequalities.

\svm{} is used because it learns a linear functions to fit the input rankings.
Therefore, the learned weights can be used directly as weights for the soft constraints.
For \svm{} a robust and efficient software system is available.